\documentclass{article}
\usepackage[T1]{fontenc}
\usepackage{changepage}
\usepackage{indentfirst}

\begin{document}

\title{Should gender-affirming care be provided to transgender young people?}

\author{Guillermo Wajner}

\date{August 2022}

\setlength{\parindent}{0.5in}
\setlength{\parskip}{0.2in}

\maketitle

\section{Introduction}

In recent years, there has been an outburst in popular media regarding the coverage of transgender people’s experience. Alongside this phenomenon has come a rapid increase in the number of people identifying with the label, as around 1.4\% of 13 to 17-year-olds do so, compared to approximately 0.5\% of adults (Herman et al., 2022). While some attribute this occurrence to people simply having the proper vocabulary to express their feelings and people being less concerned about almost certain social backlash, others believe that it may be rooted in confusion and/or working as a form of trend.\par

Besides, those previously unaware of the situation have been quick to come up with questions and outline potential issues. One of the main concerns that has risen, and which has quickly gained importance in the political world, is how to deal with these teenagers and kids who report experiencing gender dysphoria (GD), because even in the medical scene it seems to not be that clear due to how recent and ever-changing the situation is.\par

It is to be noted that not all forms of gender transition are medical, and even more so for teenagers, however it is mainly medical transitions which have sparked scepticism and/or criticism. Therefore, the rest of the article will focus on this issue.

\section{Causes}

Transgender healthcare is not a heavily researched field, and most studies regarding this were conducted with either minimal sample sizes, from which it is not possible to accurately extrapolate, or use overall outdated methods and concepts, which renders a lot of older studies useless in the modern context of medicine. Parents and doctors alike may struggle to keep up with such a dynamic and fluid research field. Consequently, those who may be trying to do the best for their children or their patient, respectively, may be unintendedly doing more harm than good.\par

Kid patients who seek for gender-affirming medical care typically come from an early, unsought diagnosis of gender dysphoria. This diagnosis is mainly based on a professional assessment of primary reports from the kid, and doesn't come from the patient themselves expressing the possibility of experiencing gender dysphoria. This, however, is not the case for older teens, who more often than not have already made the assessment themselves and are merely looking for a psychological professional to give them the diagnosis in order to gain access to the appropriate treatment. In this case, patients' claims are not heavily questioned and the psychologist's role becomes more of a mere bureaucratic tool.\par

This is a valid reason for concern, considering the possibility that these self-diagnoses in teenagers may be rooted in things other than actual clinical gender dysphoria. For instance, one of the most prevalent alternative concepts is that coined by Lisa Littman (2018): “rapid-onset gender dysphoria”, which proposes that late stage appearances of gender dysphoria, especially in assigned-female-at-birth cases, “occur in the context of belonging to a peer group where one, multiple, or even all of the friends have become gender dysphoric and transgender-identified during the same timeframe” (p. 1). This notion was further popularised and mainstreamed by Abigail Shrier (2020) in her book Irreversible damage: The transgender craze seducing our daughters, and is currently mostly prevalent in “gender critical feminist” discourse in the United Kingdom.

\newpage

\section{Consequences}

Providing gender-affirming care to teenagers may have varying outcomes on a patient to patient basis, and proper ethical considerations must be made when providing it. These are succinctly outlined by Kimberly et al. (2018) in their state-of-the-art review in the following way: \par

\begin{quote}
    
\indent Ethical considerations in gender-affirming care for transgender and gender-nonconforming youth span concerns about meeting the obligations to maximize treatment benefit to patients (beneficence), minimizing harm (nonmaleficence), supporting autonomy for pediatric patients during a time of rapid development, and addressing justice, including equitable access to care for transgender and gender nonconforming youth. (p. 1)

\end{quote}\par

Among the main causes of concern lies the possibility of regret, as there are changes produced by forms of gender-affirming care that are very hard to undo, especially regarding masculinising hormone therapy’s effect on voice or feminising hormone therapy’s breast growth. It is arguable that at such an early stage in someone's life, making such important decisions and some that will vastly affect their future prospects is not a task they should be executing and may be detrimental in the long term. Therefore, people critical of these methods claim that doctors and psychologists alike are doing a disservice to the patients who come into their office looking for help, full of doubts and concerns and receive no more than encouragement and nice words.\par

Transgender people are known to be much more prone to having suicidal behaviour (Toomey et al., 2018). For transgender youth with gender dysphoria who actually go through with medical transition, studies show very serious improvements in issues such as suicide ideation, suicide attempt and nonsuicidal self-injury, which go from 81\% to 39\%, 16\% to 4\% and 52\% to 18\% respectively (Kuper et al., 2020). Furthermore, patients also typically report great improvements in quality of life, and a smaller degree of GD and general mental distress (Nguyen et al., 2018).\par

Patients who choose not to go through with medical transition but to still present as the gender they identify with are more likely to face rash discrimination in their daily lives, as prejudice is harsher and happens more often towards individuals whose physical appearance diverges more from their gender presentation. It is definitely not as though one who presents in discordance with societal expectations of their sex will be exempt from bigotry just by not undergoing any form of medical intervention.

\section{Global Perspective}

As stated before, there are several perspectives which oppose gender-affirming transgender care, and perspectives on the issue may vary wildly. While some do come from a place of exclusive concern regarding the youth, most come from an all encompassing rejection of gender transition, and some may even stem from an argument in favour of the abolition of gender itself.\par

In the first place, when it comes to youth focused perspectives, most prevalent is Littman’s aforementioned rapid-onset gender dysphoria hypothesis. However, Littman’s methods have been criticised on several occasions (Farley et al., 2020; Ashley, 2020) and should be “best understood as an attempt to mobilise scientific language to circumvent mounting evidence in favour of gender affirmation” (Restar, 2020, p. 1). Littman uses a pathologising and stigmatising framework and solely surveys parents in forums dedicated to sharing ideas against gender-affirming care. The strikingly flawed methods employed in Littman’s studies have naturally resulted in her interpretation not being endorsed or defended by any medical institutions, in spite of the amount of people who may find this idea to be a compelling explanation of recent statistical trends.\par

Regarding the claims that minors are simply too young to make life altering decisions such as this one, and the concerns of regret regarding said decisions, it must be noted that sources such as the 2015 U.S. Transgender Survey have found that some kind of detransition was experienced by around 8\% of respondents, of which 62\% declared only doing so because of either financial issues or social or parental pressure (James et al., 2016). However, often cited sources claim much higher rates of detransition, from which Steensma et al.'s (2011) qualitative study stands out. In this study, the claim that 84\% of kids who experience gender dysphoria are desisters is present, which means they eventually “grow out” of these feelings. However, this study has been greatly discredited because of its lack of methodical rigorously, as it fails to distinguish between kids with consistent, persistent, and insistent gender dysphoria, kids who only socially transition and kids who just acted in a less gender-conforming manner. Furthermore, the study also failed to do the follow-up on more than 45\% of the kids involved and simply jumped to the conclusion that they belonged in the desisters group, reasons unknown.\par

Arguments in favour of the idea of gender-affirming care for the youth rely primarily on the principles of beneficence and nonmaleficence, because as stated by Kuper et al. (2020) and Nguyen et al. (2018), there are proven benefits for the patient, including evident improvements both in quality of life and overall mental health. Another argument in favour of the notion is that even if the premise that “children are too young for making decisions of this sort” is accepted, this in no way contradicts the prescription of hormone blockers for gender dysphoric children, as not doing so could still result (and more probably so) in the patient’s body going through unwanted, critical changes, only for them to have to deal with more problems later in life. Failing to provide gender-affirming care early on will only make them more likely to face rash discrimination in their lives, as prejudice is harsher and happens more often towards individuals whose physical appearance diverges more from their gender presentation, and it is undeniable that when starting transition at a later stage there are changes produced by puberty that the patient may despise which are unchangeable without more intrusive surgical procedures. Besides, there is no real danger in ascribing hormone blockers, as these only postpone the effects of puberty until the child either is old enough to go through full hormone replacement therapy or finds they are comfortable going through natural puberty.

\section{National Perspective}

In the country of Uruguay, it is true that although media is much less outspoken about transgender issues, the national medical field is almost ubiquitously in favour of providing the transgender youth with appropiate councelling and medical care. Even more so, hormone replacement therapy is provided as a free service as a part of the country's public healthcare system. Besides, Uruguay was the first country in the world to have a government conducted census exclusively for transgender people done in 2016, and have since remained on the forefront of the issue. However, on a more societal level, things are definitely less appealing. Transgender people in Uruguay are possibly the most marginalized group one can think of, as among them 83\% don’t complete highschool, 30\% are unemployed, 65\% are working informal jobs, mainly in the sex industry, and 88\% claim having experienced some form of discrimination.

\section{Personal Perspective}

On a personal level, I believe that the evidence clearly suggests that we should be wholly in favour of providing transgender children and teenagers with the care they deserve. We should strive to depoliticise this issue and make sure we do what is best for them, as it concerns me that in spite of scientific efforts and achievements, people seem to be always looking for outlandish reasons to discredit methods of care already proven efficient.

\section{Courses of Action}

The main course of action that must be taken to ensure proper decisions are made concerning the matter in question, is to actively share information to those who may need it, and to help them distinguish between fact and falsehood. In the age of modern communication, where misinformation runs rampant, it is us who are the only force against it, and must therefore push back whenever possible. 

\section{Conclusions}

After taking different perspectives into consideration, it becomes rather clear that the stance taken by medical associations all over the world is heavily backed up by academic research, despite some minor inner-conflicts. However, this is clearly not the case for the opposing perspective, so it is only natural that the proper measures when presented upon a young patient experiencing gender dysphoria should be providing them with a gender-affirming plan designed according to their age, while letting them further explore their identity and find out what the best way to carry on would be.

\section{Reflections}
	
When starting this investigation, I already had a quite formed opinion for which I felt comfortable arguing. I also had strong feelings regarding this issue because I’ve seen both people close to me experience hatred on the basis of being transgender, and friends of mine who held thesemisinstructed beliefs themselves. The making of this has allowed me, however, to delve deeper into ideas opposing my own, and although my overall position may not have shifted, I believe that I have still found new things to turn my gears and have gained a more profound understanding of the issue at hand.\par

Following into the future, I believe I will be much more readily prepared to defend my preheld perspective on the matter and will advocate for what I believe rightful when necessary.

\newpage

\section{Bibliography}

\begin{enumerate}

\item[-]Herman, J. L., Flores, A. R., \& O'Neill, K. K. (2022). \textit{How Many Adults and Youth Identify as Transgender in the United States?}.

\item[-]Kimberly, L. L., Folkers, K. M., Friesen, P., Sultan, D., Quinn, G. P., Bateman-House, A., ... \& Salas-Humara, C. (2018). \textit{Ethical issues in gender-affirming care for youth.} Pediatrics, 142(6).

\item[-]Littman, L. (2018). \textit{Rapid-onset gender dysphoria in adolescents and young adults: A study of parental reports.} PloS one, 13(8).

\item[-]Shrier, A. (2020). \textit{Irreversible damage: The transgender craze seducing our daughters.} Simon and Schuster.

\item[-]Toomey, R. B., Syvertsen, A. K., \& Shramko, M. (2018). \textit{Transgender Adolescent Suicide Behavior.} Pediatrics, e20174218.

\item[-]Kuper, L. E., Stewart, S., Preston, S., Lau, M., \& Lopez, X. (2020). \textit{Body Dissatisfaction and Mental Health Outcomes of Youth on Gender-Affirming Hormone Therapy.} Pediatrics, 145(4), e20193006.

\item[-]Nguyen, H. B., Chavez, A. M., Lipner, E., Hantsoo, L., Kornfield, S. L., Davies, R. D., \& Epperson, C. N. (2018). \textit{Gender-affirming hormone use in transgender individuals: impact on behavioral health and cognition.} Current psychiatry reports, 20(12), 1-9.

\item[-]Farley, L., \& Kennedy, R. M. (2020). \textit{Transgender embodiment as an appeal to thought: A psychoanalytic critique of “rapid onset gender dysphoria”.} Studies in Gender and Sexuality, 21(3), 155-172.

\item[-]Ashley, F. (2020). \textit{A critical commentary on ‘rapid-onset gender dysphoria’.} The Sociological Review, 68(4), 779-799.

\item[-]Restar, A. J. (2020). \textit{Methodological critique of Littman’s (2018) parental-respondents accounts of “rapid-onset gender dysphoria”.} Archives of Sexual Behavior, 49(1), 61-66.

\item[-]James, S., Herman, J., Rankin, S., Keisling, M., Mottet, L., \& Anafi, M. A. (2016). \textit{The report of the 2015 US transgender survey.}

\item[-]Steensma, T. D., Biemond, R., de Boer, F., \& Cohen-Kettenis, P. T. (2011). \textit{Desisting and persisting gender dysphoria after childhood: A qualitative follow-up study.} Clinical Child Psychology and Psychiatry, 16(4), 499–516.

\item[-]Ministerio de Desarrollo Social (2016). Transforma 2016 - \textit{Visibilizando realidades: avances a partir del primer censo de personas trans.}

\end{enumerate}

\end{document}