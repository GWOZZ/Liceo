\documentclass{article}
\usepackage[utf8]{inputenc}
\usepackage[spanish]{babel}
\usepackage{amsmath}
\usepackage[table,xcdraw]{xcolor}
\usepackage{siunitx}
\usepackage{float}
\usepackage{pgfplots}
\pgfplotsset{compat = newest}
\usepackage{tikz}
\usepackage{filecontents}
\usepackage{cancel}

\begin{filecontents*}{data.csv}
x,y,y1,y2,y3,y4,y5,,
0.00 ,-1.08 ,-0.44 ,1.17 ,0,0,0,0,
0.03 ,-1.00 ,-0.41 ,1.08 ,0,0,0,0,
0.07 ,-0.96 ,-0.41 ,1.04 ,0,0,0,0,
0.10 ,-0.94 ,-0.38 ,1.02 ,0,0,0,0,
0.13 ,-0.93 ,-0.37 ,1.00 ,0,0,0,0,
0.17 ,-0.91 ,-0.38 ,0.99 ,0,0,0,0,
0.20 ,-0.91 ,-0.38 ,0.99 ,0,0,0,0,
0.23 ,-0.93 ,-0.39 ,1.01 ,0,0,0,0,
0.27 ,-0.91 ,-0.38 ,0.99 ,0,0,0,0,
0.30 ,-0.93 ,-0.38 ,1.00 ,0,0,0,0,
0.33 ,-0.93 ,-0.38 ,1.00 ,0,0,0,0,
0.37 ,-0.91 ,-0.37 ,0.98 ,0,0,0,0,
0.40 ,-0.93 ,-0.35 ,0.99 ,0,0,0,0,
0.43 ,-0.73 ,-0.31 ,0.79 ,0.00 ,0.00 ,0.00 ,,
0.47 ,-0.30 ,-0.17 ,0.34 ,-0.65 ,-0.22 ,0.68 ,,
0.50 ,-0.08 ,-0.06 ,0.10 ,-0.77 ,-0.22 ,0.80 ,,
0.53 ,-0.17 ,-0.11 ,0.20 ,-0.73 ,-0.25 ,0.77 ,,
0.57 ,-0.20 ,-0.14 ,0.24 ,-0.66 ,-0.24 ,0.70 ,,
0.60 ,-0.17 ,-0.11 ,0.20 ,-0.56 ,-0.18 ,0.59 ,,
0.63 ,-0.20 ,-0.13 ,0.23 ,-0.58 ,-0.21 ,0.61 ,,
0.67 ,-0.22 ,-0.14 ,0.27 ,-0.59 ,-0.22 ,0.63 ,,
0.70 ,-0.20 ,-0.10 ,0.22 ,-0.59 ,-0.18 ,0.62 ,,
0.73 ,-0.18 ,-0.10 ,0.21 ,-0.56 ,-0.18 ,0.59 ,,
0.77 ,-0.18 ,-0.13 ,0.22 ,-0.56 ,-0.18 ,0.59 ,,
0.80 ,-0.18 ,-0.11 ,0.21 ,-0.58 ,-0.18 ,0.60 ,,
0.83 ,-0.18 ,-0.13 ,0.22 ,-0.58 ,-0.22 ,0.62 ,,
0.87 ,-0.17 ,-0.11 ,0.20 ,-0.56 ,-0.21 ,0.60 ,,
0.90 ,-0.15 ,-0.10 ,0.18 ,-0.58 ,-0.21 ,0.61 ,,
0.93 ,-0.17 ,-0.10 ,0.20 ,-0.56 ,-0.20 ,0.60 ,,
0.97 ,-0.20 ,-0.10 ,0.22 ,-0.53 ,-0.20 ,0.57 ,,
1.00 ,-0.18 ,-0.11 ,0.21 ,-0.55 ,-0.17 ,0.57 ,,
1.03 ,-0.18 ,-0.10 ,0.21 ,-0.53 ,-0.14 ,0.55 ,,
1.07 ,-0.18 ,-0.11 ,0.21 ,-0.55 ,-0.20 ,0.58 ,,
1.10 ,-0.15 ,-0.13 ,0.20 ,-0.55 ,-0.20 ,0.58 ,,
1.13 ,-0.14 ,-0.10 ,0.17 ,-0.52 ,-0.15 ,0.54 ,,
1.17 ,-0.17 ,-0.08 ,0.19 ,-0.55 ,-0.17 ,0.57 ,,
1.20 ,-0.15 ,-0.08 ,0.18 ,-0.53 ,-0.20 ,0.57 ,,
1.23 ,-0.14 ,-0.10 ,0.17 ,-0.52 ,-0.18 ,0.55 ,,
1.27 ,-0.17 ,-0.10 ,0.20 ,-0.53 ,-0.18 ,0.56 ,,
1.30 ,-0.15 ,-0.13 ,0.20 ,-0.55 ,-0.18 ,0.58 ,,
1.33 ,-0.15 ,-0.13 ,0.20 ,-0.52 ,-0.15 ,0.54 ,,
1.37 ,-0.11 ,-0.07 ,0.13 ,-0.49 ,-0.17 ,0.52 ,,
1.40 ,-0.15 ,-0.11 ,0.19 ,-0.53 ,-0.18 ,0.56 ,,
1.43 ,-0.18 ,-0.10 ,0.21 ,-0.53 ,-0.20 ,0.57 ,,
1.47 ,-0.11 ,-0.06 ,0.13 ,-0.49 ,-0.18 ,0.52 ,,
1.50 ,-0.11 ,-0.08 ,0.14 ,-0.49 ,-0.14 ,0.51 ,,
1.53 ,-0.11 ,-0.08 ,0.14 ,-0.55 ,-0.18 ,0.58 ,,
1.57 ,-0.14 ,-0.10 ,0.17 ,-0.49 ,-0.20 ,0.53 ,,
1.60 ,-0.15 ,-0.07 ,0.17 ,-0.48 ,-0.17 ,0.51 ,,
1.63 ,-0.14 ,-0.07 ,0.16 ,-0.51 ,-0.15 ,0.53 ,,
1.67 ,-0.13 ,-0.08 ,0.15 ,-0.49 ,-0.15 ,0.52 ,,
\end{filecontents*}
\begin{filecontents*}{data2.csv}
x,y
0.00 ,0.19 
0.03 ,0.17 
0.07 ,0.17 
0.10 ,0.16 
0.13 ,0.16 
0.17 ,0.16 
0.20 ,0.16 
0.23 ,0.16 
0.27 ,0.16 
0.30 ,0.16 
0.33 ,0.16 
0.37 ,0.16 
0.40 ,0.16 
0.43 ,
0.47 ,
0.50 ,0.14 
0.53 ,0.16 
0.57 ,0.15 
0.60 ,0.13 
0.63 ,0.13 
0.67 ,0.14 
0.70 ,0.13 
0.73 ,0.13 
0.77 ,0.13 
0.80 ,0.13 
0.83 ,0.13 
0.87 ,0.13 
0.90 ,0.13 
0.93 ,0.13 
0.97 ,0.13 
1.00 ,0.13 
1.03 ,0.12 
1.07 ,0.13 
1.10 ,0.12 
1.13 ,0.11 
1.17 ,0.12 
1.20 ,0.12 
1.23 ,0.11 
1.27 ,0.12 
1.30 ,0.12 
1.33 ,0.12 
1.37 ,0.10 
1.40 ,0.12 
1.43 ,0.12 
1.47 ,0.10 
1.50 ,0.10 
1.53 ,0.11 
1.57 ,0.11 
1.60 ,0.11 
1.63 ,0.11 
1.67 ,0.11 
\end{filecontents*}
\begin{filecontents*}{data3.csv}
y,x
0.106,0
0.091,0.03
0.084,0.07
0.08,0.1
0.078,0.13
0.076,0.17
0.076,0.2
0.079,0.23
0.076,0.27
0.078,0.3
0.078,0.33
0.075,0.37
0.077,0.4
0.051,0.5
0.05,0.53
0.043,0.57
0.03,0.6
0.034,0.63
0.037,0.67
0.034,0.7
0.031,0.73
0.031,0.77
0.032,0.8
0.034,0.83
0.031,0.87
0.032,0.9
0.031,0.93
0.029,0.97
0.029,1
0.027,1.03
0.03,1.07
0.03,1.1
0.025,1.13
0.028,1.17
0.028,1.2
0.026,1.23
0.028,1.27
0.029,1.3
0.026,1.33
0.022,1.37
0.028,1.4
0.029,1.43
0.023,1.47
0.022,1.5
0.028,1.53
0.024,1.57
0.022,1.6
0.024,1.63
0.023,1.67
\end{filecontents*}
\definecolor{yellowishy}{RGB}{221,170,51}
\definecolor{redishy}{RGB}{187,85,102}
\definecolor{blueishy}{RGB}{0,68,136}

\title{Análisis de Video de Colisión}
\author{Martín Vázquez, Matías Vicente, Guillermo Wajner}
\date{Setiembre 2022}

\begin{document}

\maketitle

\section{Introducción}

El presente informe ha sido desarrollado en base a los datos recabados del análisis de video realizado en la aplicación “Tracker” con el propósito de analizar el fenómeno de colisión entre dos bolas de billar.

\section{Objetivo}

Analizar la conservación de cantidad de movimiento y la variación de la energía mecánica en un choque a partir de una filmación de una colisión entre dos bolas de billar.

\section{Marco Teórico}

Entiéndase por cantidad de movimiento o momento lineal al producto de la masa y velocidad de un cuerpo.
\begin{flalign}
    \vec{p}=m\times\vec{v}
\end{flalign}
Considérese choque a toda colisión entre dos cuerpos los cuales por un momento breve de tiempo entran en contacto y se ejercen fuerzas mutuamente. En todo choque se da que la cantidad de movimiento del sistema no varía desde el instante previo al instante posterior del mismo ($\varDelta \vec{p}=0$)
\\
\\
Un choque se puede clasificar según la variación de energía total del sistema de la siguiente forma:
\\
\\
Si se conserva la energía cinética del sistema ($\varDelta E_k=0$) en la colisión se trata de un choque elástico.\\
Si existe una pérdida de energía cinética ($\varDelta E_k<0$) en la colisión, existen dos posibilidades: cuando los cuerpos continúen desplazándose juntos  es completamente inelástico, y cuando no simplemente es inelástico.

\section{Procedimiento}

En primera instancia, en Tracker se define el diámetro de los objetos sometidos a estudio (bolas de billar)  \qty{51.75}{mm} como referencia de medida. A continuación, se fija una masa puntual en el centro de la bola blanca para después por medio de ella realizar el rastreo de esta hasta que una de las bolas finalice su recorrido en el trayecto estudiado. Finalmente, se toma la magnitud de la velocidad para ambas bolas a lo largo del tiempo, acompañado de la descomposición en el eje x y en el eje y de la velocidad. A raíz de esa descomposición, se halla el ángulo con que se desplazaron ambas bolas en el momento previo y después del choque. 

\section{Datos Obtenidos}

Los siguientes gráficos fueron realizados con las medidas obtenidas:

\begin{figure}[h!]
\centering
\begin{tikzpicture}
\begin{axis}[
y tick label style={
        /pgf/number format/.cd,
            fixed,
            fixed zerofill,
            precision=1,
        /tikz/.cd
    },
    x tick label style={
        /pgf/number format/.cd,
            fixed,
            fixed zerofill,
            precision=2,
        /tikz/.cd
    }
    xmin = 0, xmax = 2.25,
    ymin = -1.5, ymax = 1.5,
    xtick distance = 0.25,
    ytick distance = 0.5,
    xlabel={$t$ (s)},
    ylabel={$v$ (m/s)},
    grid = both,
    minor tick num = 1,
    major grid style = {lightgray!80},
    minor grid style = {lightgray!25},
    width = \textwidth,
    height = 4.8cm,
    legend cell align = {left},
    legend pos = north east
]
 
\addplot[yellowishy, mark = *, mark size=0.04cm] table [x = {x}, y = {y}, col sep=comma] {data.csv};
 
\addplot[redishy, mark = *, mark size=0.04cm] table [x ={x}, y = {y1}, col sep=comma] {data.csv};
 
\addplot[blueishy, mark = *, mark size=0.04cm] table [x = {x}, y = {y2}, col sep=comma] {data.csv};
 
\legend{
    $\vec{v_x}/\hat{\imath}$, 
    $\vec{v_y}/\hat{\jmath}$,
    $v$
}
 
\end{axis}
\end{tikzpicture}
\label{fig:vb}
\caption{Velocidad de la Bola Blanca}
\end{figure}

\begin{figure}[H]
\centering
\begin{tikzpicture}
\begin{axis}[
y tick label style={
        /pgf/number format/.cd,
            fixed,
            fixed zerofill,
            precision=1,
        /tikz/.cd
    },
    x tick label style={
        /pgf/number format/.cd,
            fixed,
            fixed zerofill,
            precision=2,
        /tikz/.cd
    }
    xmin = 0, xmax = 2.25,
    ymin = -1.5, ymax = 1.5,
    xtick distance = 0.25,
    ytick distance = 0.5,
    xlabel={$t$ (s)},
    ylabel={$v$ (m/s)},
    grid = both,
    minor tick num = 1,
    major grid style = {lightgray!80},
    minor grid style = {lightgray!25},
    width = \textwidth,
    height = 4.8cm,
    legend cell align = {left},
    legend pos = north east
]
 
\addplot[yellowishy, mark = *, mark size=0.04cm] table [x = {x}, y = {y3}, col sep=comma] {data.csv};
 
\addplot[redishy, mark = *, mark size=0.04cm] table [x ={x}, y = {y4}, col sep=comma] {data.csv};
 
\addplot[blueishy, mark = *, mark size=0.04cm] table [x = {x}, y = {y5}, col sep=comma] {data.csv};
 
\legend{
    $\vec{v_x}/\hat{\imath}$, 
    $\vec{v_y}/\hat{\jmath}$,
    $v$
}

\end{axis}
\end{tikzpicture}
\label{fig:vn}
\caption{Velocidad de la Bola Naranja}
\end{figure}

\section{Procesamiento de Datos}

Al observar las figuras 1 y 2 es posible observar que los valores más próximos al momento del choque donde se dan los cambios de velocidad, se distancian de sus valores adyacentes. Esto se debe a que al no darse la colisión en un cuadro de video exacto, los valores próximos no resultan representativos del hecho. Por esto fue tomada la decisión de excluir dichos puntos de datos.
\\
\\
Para analizar la cantidad de movimiento, fue empleado como valor de la masa de una bola de billar \qty{0,156}{\kilogram}. Utilizando la Ecuación 1 y propiedades de vecotres, es posible despejar $\vec{p}$ de la siguiente forma:
\\
\begin{flalign*}
    &\begin{cases}
      \vec{p}\hphantom{.}=\hphantom{.}\vec{p_x}+\vec{p_y}\\
      \vec{p_\alpha}\hphantom{.}=\hphantom{.}m\times\vec{v_\alpha}\\
      \vec{v_\alpha}\hphantom{.}=\hphantom{.}\vec{v_{\alpha1}}+\vec{v_{\alpha2}}
    \end{cases}
    \Longrightarrow\hphantom{.}\vec{p}\hphantom{.}=\hphantom{.}m\times(\vec{v_{x\text{B}}}+\vec{v_{x\text{N}}})+m\times(\vec{v_{y\text{B}}}+\vec{v_{y\text{N}}})&
\end{flalign*}
\begin{flalign*}
    &\Longrightarrow\hphantom{.}\mid\vec{p}\mid\hphantom{.}=\hphantom{.}\sqrt{m^2\times(\vec{v_{x\text{B}}}+\vec{v_{x\text{N}}})^2+m^2\times(\vec{v_{y\text{B}}}+\vec{v_{y\text{N}}})^2}&\\
    &\Longrightarrow\hphantom{.}\mid\vec{p}\mid\hphantom{.}=\hphantom{.}m\times\sqrt{(\vec{v_{x\text{B}}}+\vec{v_{x\text{N}}})^2+(\vec{v_{y\text{B}}}+\vec{v_{y\text{N}}})^2}
\end{flalign*}

\hphantom{.}\\
A continuación se encuentra el gráfico de este valor $\mid\vec{p}\mid$ en función del tiempo utilizando los valores de $\vec{v_x}$ y $\vec{v_y}$ obtenidos:

\begin{figure}[h!]
\centering
\begin{tikzpicture}
\begin{axis}[
y tick label style={
        /pgf/number format/.cd,
            fixed,
            fixed zerofill,
            precision=1,
        /tikz/.cd
    },
    x tick label style={
        /pgf/number format/.cd,
            fixed,
            fixed zerofill,
            precision=2,
        /tikz/.cd
    }
    xmin = 0, xmax = 2.25,
    ymin = 0, ymax = 0.3,
    xtick distance = 0.25,
    ytick distance = 0.05,
    xlabel={{$t$} (\unit[per-mode = symbol]{\second})},
    ylabel={{$p$} (\unit[per-mode = symbol]{\kilogram\metre\per\second})},
    grid = both,
    minor tick num = 1,
    major grid style = {lightgray!80},
    minor grid style = {lightgray!25},
    width = \textwidth,
    height = 4.8cm,
    legend cell align = {left},
    legend pos = north east
]
 
\addplot[redishy, mark = *, mark size=0.04cm] table [x = {x}, y = {y}, col sep=comma] {data2.csv};
 
\legend{
    $\vec{p}$
}
 
\end{axis}
\end{tikzpicture}
\label{fig:cm}
\caption{Cantidad de Movimiento del Sistema}
\end{figure}

\noindent Observando esta gráfica es visible que en los instantes anteriores y posteriores a la colisión la cantidad de movimiento no sufre variaciones externas a las despreciables perturbaciones atribuibles a errores de experimentación.

\newpage\noindent Para averiguar que tipo de choque es el analizado, fue graficada la energía cinética del sistema en funcion del tiempo utilizando la siguiente ecuación:

\begin{equation*}
    E_k=\frac{m\times(v_B^2+v_N^2)}{2}
\end{equation*}

\begin{figure}[h!]
\centering
\begin{tikzpicture}
\begin{axis}[
y tick label style={
        /pgf/number format/.cd,
            fixed,
            fixed zerofill,
            precision=1,
        /tikz/.cd
    },
    x tick label style={
        /pgf/number format/.cd,
            fixed,
            fixed zerofill,
            precision=2,
        /tikz/.cd
    }
    xmin = 0, xmax = 2.25,
    ymin = 0, ymax = 0.3,
    xtick distance = 0.25,
    ytick distance = 0.05,
    xlabel={{$t$} (\unit[per-mode = symbol]{\second})},
    ylabel={{$E$} (\unit[per-mode = symbol]{\newton})},
    grid = both,
    minor tick num = 1,
    major grid style = {lightgray!80},
    minor grid style = {lightgray!25},
    width = \textwidth,
    height = 4.8cm,
    legend cell align = {left},
    legend pos = north east
]
 
\addplot[yellowishy, mark = *, mark size=0.04cm] table [x = {x}, y = {y}, col sep=comma] {data3.csv};
 
\legend{
    $E_k$
}
 
\end{axis}
\end{tikzpicture}
\label{fig:cm}
\caption{Energía Cinética del Sistema}
\end{figure}

\noindent A partir de esta gráfica es visible que se trata de un choque inelástico ya que existe un salto significativo en el valor de $E_k$ del momento previo al choque al posterior el cual corresponde a una perdida de energía.  

\section{Conclusión}

Finalizado el análisis del choque en cuestión, es posible concluir que efectivamente la cantidad de movimiento se conserva un instante antes y después de la colisión. Además, fue posible averiguar que se trataba de un choque inelástico, a pesar de que si el sistema fuese perfecto (esferas perfectas en el vacío), el choque sería elástico.

\end{document}
