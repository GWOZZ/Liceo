\documentclass{article}
\usepackage[utf8]{inputenc}
\usepackage{graphicx}
\usepackage{float}
\usepackage{tabularx}
\usepackage{siunitx}

\begin{document}

\title{Title of Your Document}
\author{Your Name}
\date{\today}
\maketitle

\section{OBJETIVO}
El objetivo de esta actividad es estudiar las propiedades de los lípidos, específicamente del aceite, manteca y grasa, y observar cómo reaccionan en diferentes situaciones.

\section{MARCO TEÓRICO}
Para dar mayor entendimiento a este informe es necesario definir previamente algunos conceptos:

\begin{itemize}
    \item Lípidos: Biomoléculas orgánicas insolubles en agua pero solubles en solventes orgánicos no polares. Tienen una estructura rica en carbono e hidrógeno, lo que les confiere propiedades físicas y químicas particulares que analizaremos y comprobaremos en el presente práctico.
    
    \item Grasas: Mezclas de lípidos simples (formados por carbono, hidrógeno y oxígeno) que forman parte de la familia de los glicéridos. En lo que concierne a su estructura química, la molécula está normalmente compuesta por tres ácidos grasos, siendo la mayor parte de estos saturados (no presentan enlaces dobles).
    
    \item Aceites: Al igual que las grasas, estos son mezclas de lípidos simples cuyo alcohol constituyente es el glicerol. En su estructura química existe una predominancia de ácidos grasos monoinsaturados o poliinsaturados.
    
    \item Mantecas: Grasas caracterizadas por estar compuestas principalmente por ácidos grasos con un bajo número de átomos de carbono.
\end{itemize}

\section{MATERIALES}
\begin{itemize}
    \item Tubos de ensayo.
    \item Probeta de 10 mL.
    \item Papel de filtro.
\end{itemize}

\section{SUSTANCIAS Y/O SOLUCIONES}
\begin{itemize}
    \item Aceite comestible.
    \item Agua.
    \item Disán.
    \item Grasa animal.
    \item Manteca.
\end{itemize}


\section{PROCEDIMIENTO}
\begin{enumerate}
    \item Colocar en 2 tubos de ensayo 20 gotas de aceite comestible. Agregar en uno de los tubos 3 mL de agua y en otro 3 mL de disán. Agitar cada tubo y anotar las observaciones. Completar el cuadro.
    \item Repetir el paso 1 utilizando las otras muestras de lípidos (manteca y grasa). Completar el cuadro.
    \item Calentar los tubos de la parte 1 y 2 a baño María. Observar y completar el cuadro.
    \item Tome 3 tubos y coloque una muestra de manteca, cera y grasa (uno en cada uno). Caliente los mismos a baño María y anote la temperatura de fusión aproximada de los mismos. Complete el cuadro.
    \item Determine la masa de una probeta de 10 mL. Complete la misma con 10 ml de aceite y determine la masa total.
    \item A partir de los datos obtenidos, calcule la densidad del aceite.
    \item Tome un trocito de papel de filtro y deje caer una gota de aceite. Observe. Repita lo anterior utilizando las diferentes muestras de lípidos.
\end{enumerate}

\section{Procesamiento de Datos}

\begin{table}[H]
    \begin{tabular}{|p{0.2\textwidth}|p|p|p|}
    \hline
    Propiedad                                   & Aceite                           & Manteca                                   & Grasa                                     \\ \hline
    \raggedright Estado físico a temperatura ambiente        & Líquido                          & Sólido                                    & Sólido                                    \\ \hline
    \raggedright Solubilidad en agua a temperatura ambiente  & Insoluble (Dos fases líquidas)   & Insoluble (Una fase sólida y una líquida) & Insoluble (Una fase sólida y una líquida) \\ \hline
    \raggedright Solubilidad en agua caliente                & Insoluble (Dos fases líquidas)   & Insoluble (Una fase sólida y una líquida) & Insoluble (Una fase sólida y una líquida) \\ \hline
    \raggedright Solubilidad en disán a temperatura ambiente & Soluble (Disolución instantánea) & Soluble (Disolución lenta)                & Soluble (Disolución lenta)                \\ \hline
    \raggedright Solubilidad en disán caliente               & Soluble (Disolución instantánea) & Soluble (Disolución rápida)               & Soluble (Disolución rápida)               \\ \hline
    \raggedright Punto de fusión                             & $< 20$ $^{\circ}$C               & 29 $^{\circ}$C                            & 39 $^{\circ}$                             \\ \hline
    \raggedright Acción sobre el papel de filtro             & Mancha transparente              & Mancha grasosa                            & Mancha grasosa                            \\ \hline
    \end{tabular}
\end{table}

\section{CONCLUSIONES}
En esta actividad, hemos estudiado las propiedades de diferentes lípidos: el aceite, la manteca y la grasa. Observamos que el aceite es líquido a temperatura ambiente y no se disuelve en agua, pero sí en disán. Al calentarse levemente, el aceite se vuelve menos viscoso, pero no afecta su estado de agregación. Por otro lado, la manteca y la grasa son sólidas a temperatura ambiente y no se disuelven en agua ni en disán. Al calentarse, cada uno presenta un punto de fusión diferente.

A su vez, es posible afirmar que la presencia de insaturaciones en las cadenas carbonadas de los ácidos grasos incide en el punto de fusión. Aquellos lípidos con presencia de insaturaciones en su estructura (aceites) tendrán un menor punto de fusión que aquellos que no las tienen (grasas). También concluimos que el punto de fusión es proporcional a la longitud de las cadenas carbonadas de los ácidos grasos constituyentes, ya que la manteca presenta cadenas más cortas, y considerando que las ceras (no contempladas en este práctico), poseen un alto punto de fusión y cadenas carbonadas largas.

\section{ACTIVIDADES COMPLEMENTARIAS}
\begin{enumerate}
    \item ¿Cuál es la composición química de aceites, grasas y ceras?

    \begin{itemize}
        \item Grasas: Mezclas de lípidos simples (formados por carbono, hidrógeno y oxígeno) que forman parte de la familia de los glicéridos. En lo que concierne a su estructura química, la molécula está normalmente compuesta por tres ácidos grasos, siendo la mayor parte de estos saturados (no presentan enlaces dobles).
        
        \item Aceites: Al igual que las grasas, estos son mezclas de lípidos simples cuyo alcohol constituyente es el glicerol. En su estructura química existe una predominancia de ácidos grasos monoinsaturados o poliinsaturados.

        \item Ceras: Son lípidos simples formados como producto de la reacción de ácidos grasos con un alto número de átomos de carbono (24 - 36 átomos) y un alcohol que también presenta una gran presencia de átomos de carbono en su estructura (16 - 36 átomos).
    \end{itemize}
    
    \item ¿Qué son las grasas trans y por qué debemos evitarlas?

    Las grasas trans son un tipo de grasa insaturada que se produce mediante un proceso llamado hidrogenación parcial, que convierte los aceites líquidos en grasas sólidas a temperatura ambiente. Se encuentran principalmente en alimentos procesados y productos horneados comerciales, como galletas, pasteles, papas fritas y margarina.
    
    Estas grasas son particularmente problemáticas porque tienen efectos negativos en la salud. A diferencia de las grasas insaturadas saludables, como las presentes en el aceite de oliva y los aguacates, las grasas trans aumentan los niveles de colesterol LDL (el colesterol "malo") y disminuyen los niveles de colesterol HDL (el colesterol "bueno"), lo que aumenta el riesgo de enfermedades cardiovasculares.
    
    Además, las grasas trans también se han asociado con otros problemas de salud, como la inflamación, la resistencia a la insulina y el aumento del riesgo de desarrollar diabetes tipo 2. También pueden afectar negativamente la salud arterial, promoviendo la formación de placas de ateroma y aumentando el riesgo de accidentes cerebrovasculares.
    
    \item ¿Qué es mejor incluir en nuestras dietas: margarinas o manteca? ¿Por qué?
    El consumo de manteca por sobre el de la margarina es signo de una buena dieta debido a que la margarina sufre una serie de procesos de industrialización que configuran un producto de carácter hidrogenado que, si es consumido reiteradamente, puede ser dañino para la salud. El proceso de hidrogenación puede generar isómeros trans, que se han asociado con efectos negativos para la salud cardiovascular. Cuanto más hidrogenada sea la margarina, mayor será su contenido de grasas trans. Es necesario aclarar que este no es el caso para absolutamente todos, ya que una persona intolerante a la lactosa no podrá consumir manteca.
\end{enumerate}
\end{document}