	\documentclass{article}
	\usepackage[spanish]{babel}
	\usepackage{graphicx}
	\usepackage{siunitx}
	\usepackage{amsmath}
	\usepackage{lmodern, amssymb, amsfonts}
	\usepackage{mathtools}
	\usepackage[T1]{fontenc}
	\usepackage{float}
	\usepackage[hyphens]{url}

	\title{Planes Sociales y Dependencia Estatal en el Uruguay
	}
	\author{V. Angeloro, P. Bibiloni, J. Quiroga, S. Suárez, M. Vázquez y G. Wajner}
	\date{Junio 2023}
	\setlength{\parskip}{10pt}

\begin{document}

\begin{titlepage}
	
	\centering
	
	\rule{\textwidth}{1pt}
	\vspace{2pt}\vspace{-\baselineskip}
	\rule{\textwidth}{0.4pt}

	\vspace{0.1\textheight}
	
	{\Huge Planes Sociales y Dependencia}\\[0.5\baselineskip]
	{\Huge Estatal en el Uruguay}
	
	\vspace{0.025\textheight}
	
	\rule{0.3\textwidth}{0.4pt}
	
	\vspace{0.037\textheight}
	{\Large \textsc{V. Angeloro, P. Bibiloni, J. Quiroga,}}

  \vspace{0.1cm}

  {\Large \textsc{S. Suárez, M. Vázquez y G. Wajner}}

  \vfill
	
  \large\textsc{E.E. y S. - Junio 2023}

  \vspace{-0.2cm}

  \large\textsc{Primera Prueba Pacrial 3° BD}

  \vspace{-0.2cm}

  \large\textsc{Prof. M. García}

  \vspace{0.05\textheight}

	\includegraphics[width=0.3\textwidth]{logo.jpg}
	
  \vspace{0.08\textheight}

	\rule{\textwidth}{0.4pt}
	
	\vspace{2pt}\vspace{-\baselineskip}
	
	\rule{\textwidth}{1pt}
	
\end{titlepage}

\section{Planteamiento del tema}

¿El aumento de los planes sociales significa mayor dependencia estatal de los más vulnerables? ¿De qué tiene que ir acompañado un plan social para no generar una dependencia prolongada del beneficiario?

\section{Justificación y significación social del tema}

Los planes sociales, son una cuestión que —principalmente en países en vías de desarrollo— está presente en la línea conductora que guía a los gobiernos en sus políticas públicas. Su correcta implementación es una parte importante de la fomentación del desarrollo social y económico de un país; sin embargo, esta prestación puede funcionar también como forma de conseguir réditos político-electorales, ya que es un medio por dónde el estado le puede dar prestaciones a un ciudadano de forma perpetua. Esto puede provocar en el beneficiario incertidumbre sobre qué puede suceder con sus beneficios al cambiar el color político del gobierno. Es aquí donde nacen preguntas sobre qué es lo que debe acompañar a un plan para que justamente el Estado no se encuentre en un constante estado de prestación, sino que incentive a los beneficiarios a salir de la necesidad de la prestación y que así alcancen la autosuficiencia económica.

Este es un tema de gran sensibilidad social por lo anteriormente dicho; también porque atañe derechos fundamentales de la ciudadanía, como el derecho a la vida, la educación y la vivienda digna, consagrados en el Art. 7 de la Constitución. En esto se basa parte de la presente investigación, en la forma en que son asegurados estos derechos a la hora de implementar una prestación. Un análisis de esta índole nos ayudaría a reunir conocimiento sobre cuáles son las metodologías más adecuadas para desarrollar un plan social y de este modo asegurar los derechos correspondientes al beneficiario sin perjudicar las arcas estatales.

A pesar de la clara relevancia del tema planteado, existe una escasez internacional de investigaciones y publicaciones sobre el mismo. Existen antecedentes , podemos concluir que no se han hecho numerosas investigaciones hasta ahora con relación con nuestro tema: planes sociales y dependencia estatal en Uruguay. Gracias a esto, nuestro planteamiento del tema es significativo.

\section{Marco Teórico}

Desde la teoría política del bienestar, se entiende por gobierno una entidad que debe velar por los derechos de los ciudadanos que integran su sociedad y asegurarles una vida digna. La protección social se encuentra dentro de los derechos humanos dictados en el Pacto Internacional de Derechos Económicos, Sociales y Culturales de la Organización de las Naciones Unidas.	

En la República Oriental del Uruguay esta obligación comprende no solo a los ciudadanos de la República, si no a todo residente del territorio. Cuando hablamos de residentes, nos referimos tanto a ciudadanos (naturales o legales) como a no ciudadanos. Existen dos tipos establecidos por la ley 18.250 (Ley de Migraciones): residentes permanentes, y residentes temporarios.

Los primeros, acorde al art. 32 de la ley, son todos aquellos que siendo extranjeros ingresan al país con el objetivo de establecerse definitivamente y que cumplan algunos de los siguientes requisitos: 1) Los cónyuges, concubinos, padres, hermanos y nietos de uruguayos bastando que acrediten dicho vínculo; 2) Los nacionales de los Estados Partes del MERCOSUR y Estados Asociados que acrediten dicha nacionalidad.

Por otro lado, en el art. 34 de dicha ley se define al residente temporario como una persona extranjera que ingresa al país a desarrollar una actividad por un tiempo determinado.

Volviendo al concepto inicial en dónde describimos el papel del gobierno en la acción de salvaguardar los derechos y asegurar la vida digna de los ciudadanos es donde surge la noción de plan social. Quizás el más antiguo precedente histórico de un sistema formal de asistencia a los vulnerables sostenido por impuestos, es aquel de las Leyes de Pobres Isabelinas de 1601. Esta surge a razón de una recesión económica, desempleo y hambruna generalizada. Las Leyes otorgaban a gobiernos locales el poder de recaudar impuestos para construir y mantener asilo para ancianos, y proporcionar ayuda financiera a pobres dignos, ancianos, discapacitados y otros.

En nuestro país, la formación del sistema de  seguridad social también fue en virtud de situaciones de crisis y recesión económica a principios del siglo XX. A pesar de ello, la primera legislación en materia jubilación fue en 1896 con la creación de la caja escolar, acompañada de las normas ya implantadas en el Estado Oriental en la creación de la primera Constitución  de 1830.

El Banco de Previsión Social (BPS) surge en 1967 como la primera institución con la iniciativa de intervención estatal y el objetivo principal de brindar protección social a los trabajadores uruguayos y sus familias. Hacia fines de la década del cincuenta se comenzó a discutir a nivel parlamentario la idea de centrar la administración de los beneficiarios sociales existentes. En 1958, diversos sectores políticos concretaron la propuesta en la creación del Banco de Previsión Social. Hoy día, la misión principal del BPS es brindar servicios para asegurar a la ciudadanía la cobertura de las contingencias sociales y óptima gestión de los recursos, garantizando la eficiencia, eficacia y equidad del sistema seguridad social en el marco de las responsabilidades constitucionales asignadas a la institución. El BPS es uno de los responsables de la recaudación y distribución de las contribuciones al sistema de seguridad social. Además, administra y controla el pago de las prestaciones a los trabajadores y las familias, como las jubilaciones, las pensiones y subsidios por invalidez, viudez, enfermedad, maternidad, desempleo, entre otros.

Con mayor actualidad, el Ministerio de Desarrollo Social (MIDES) fue fundado mediante la Ley N° 17.866 aprobada el 21 de marzo de 2005 durante la presidencia de Tabaré Vázquez, parcialmente en consecuencia de la crisis económica de 2002. Esta oficina estatal, es responsable de las políticas sociales nacionales, la coordinación, articulación, seguimiento, supervisión y evaluación de los planes, programas y proyectos sociales implementados por el Poder Ejecutivo para asegurar el ejercicio de los derechos sociales y disponer una gestión participativa, transparente y con mayores niveles de compromiso social y técnico. Su propósito es mejorar las condiciones de vida y reducir la pobreza de los ciudadanos, a estos se les brinda una transferencia monetaria determinada la cual se denomina “Ingreso Ciudadano”. Crea un sistema de identificación, selección y registro para familias las cuales podrán acceder a programas sociales.

El Plan de Asistencia Nacional a la Emergencia Social (PANES) otro programa el cual fue responsabilidad del MIDES, fue temporal, lanzado en 2005 por la Ley 17869. Este tenía como misión ayudar a las familias en situación de extrema pobreza. El programa incluía transferencias monetarias condicionadas y una serie de intervenciones en varios ámbitos, tenía acceso a programas de empleo, emergencias de salud y educación, acceso a la vivienda y apoyo alimentario. Incluyó el programa Construyendo Rutas de Salidas (CRS) que incluía actividades de desarrollo personal e integración ciudadana. Este programa fue reemplazado en 2008 por el Plan de Equidad que expande la base de participantes del PANES y no hay más transferencias condicionales.

Cuando hablamos de un plan social, hablamos de una forma de expresión de la asistencia social estatal que en una primera instancia pretende mejorar la calidad de vida de ciudadanos que presentan dificultades en el acceso a derechos básicos y resolver problemas llevando a cambios sociales positivos. Esto lo hace mediante la previsión de prestaciones a cambio de las cuales se le exige al beneficiario una contraprestación.

En nuestro país existen tres principales programas que brindan apoyo económico a la ciudadanía que presenta las dificultades detalladas anteriormente: el Plan de Equidad, las Asignaciones Familiares y la Tarjeta Uruguay Social.

El plan de equidad consiste en una pensión mensual otorgada por el  Banco de Previsión Social (BPS) a mujeres embarazadas y menores o personas con discapacidad en hogares vulnerables socioeconómicamente, internados en establecimientos del Instituto del Niño y el Adolescente (INAU) o en instituciones con convenio. Los beneficiarios deben estar inscriptos y asistir a institutos de educación a excepción de los casos de discapacidad en los que esto no sea posible, tener la cantidad de controles médicos acordes a su edad, tener un control médico cada tres años si son mayores de edad con discapacidades y su titular debe residir en Uruguay. El monto de la transferencia monetaria mensual es fijo en casos de beneficiarios internados en el INAU o instituciones de convenio y aquellos que tengan incapacidades físicas o psíquicas. En el resto de los casos esto varía según el nivel educativo y la cantidad de beneficiarios por familia.

Las asignaciones familiares otorgan prestaciones económicas bimestrales por parte del BPS a familias con el fin de cubrir gastos relacionados a la crianza de los hijos y consecuentemente incentivar la asistencia a establecimientos educativos y controles de salud para embarazadas y menores. Su monto varía según los ingresos del hogar y la cantidad de hijos. También brinda asistencia materno-infantil. El BPS proporciona en Centros de Promoción Social y de Salud (CPSS) asistencia médica primaria la cual incluye control del embarazo, del recién nacido y médico pediátrico, pases a especialistas, vacunas, asistencia odontológica y de ortodoncia y asistencia social. Esto le corresponde a las embarazadas y menores de edad.  Los beneficios se otorgan desde el comienzo del embarazo hasta los 14 años en caso de asistencia a educación primaria o hasta los 18 años si se cursan estudios secundarios. En caso de discapacidad es válida una asignación simple hasta los 15 años de edad en caso de recibir pensión por invalidez y en caso de no recibir esta tendrán una asignación especial de por vida. Los padres deben residir en Uruguay y deben ser trabajadores dependientes del sector privado, jubilados o pensionistas, productores rurales con hasta 200 hectáreas, trabajadores a domicilio o en subsidios transitorios.

La Tarjeta Uruguay Social (TUS) fue creada en mayo de 2006. Es un medio de pago de transferencias monetarias dirigido a hogares, que según el Índice de Carencias Críticas (ICC), padecen de  vulnerabilidad socioeconómica extrema o a hogares de aquellos que pertenezcan a grupos específicos. Estos grupos son personas trans, usuarios de refugios del MIDES, mujeres victimas de violencia o trata de personas, personas que padecen de enfermedades crónicas o estén en situación de pobreza extrema o indigencia, embarazadas y niños de hasta tres años de edad en situación de vulnerabilidad económica o menores de un año de edad cuyas madres sean usarios de Servicios de Salud del Estado ASSE y hayan nacido en cuertas maternidades (Hospital Pereira Rossell, Hospital de Clínicas, Hospital de Bella Unión, Hospital de Artigas, Hospital de Rivera). Asiste a estos hogares cubriendo los gastos para el consumo básico de alimentos y artículos de primera necesidad como artículos de higiene personal, limpieza del hogar y vestimenta.  Los beneficiarios son exonerados  del Impuesto sobre el Valor Añadido (IVA) y se puede utilizar en la `Red de Comercios Solidarios'.

A nivel general, el concepto de dependencia estatal corresponde a un estado de un sujeto donde sus medios de suficiencia provienen de asistencia estatal, de tal modo que su subsistencia está sujeta al actuar del mismo. Esta dependencia no es necesariamente negativa. Sin embargo, su prolongación temporal tiene repercusiones económicas evidentes.

\section{Elaboración de Objetivos}

Es de especial importancia para el proyecto determinar si los planes sociales necesariamente generan una dependencia estatal y en caso de hacerlo por qué.

Como objetivos secundarios, se propone:

\begin{enumerate}
	\item Analizar las consecuencias sociales, económicas y políticas de los planes sociales. 
	\item Investigar y analizar la estructura y funcionamiento de los planes sociales existentes.
	\item Investigar si existen y cuales son las políticas actuales dirigidas a la prevención de dependencia estatal y evaluar la eficacia de estas, y elaborar posibles medidas alternativas.
	\item Recopilar datos estadísticos y realizar un análisis cuantitativo sobre la implementación y resultados de los planes sociales.
	\item De confirmarse la existencia de dependencia estatal, identificar los factores que contribuyen a la misma, en lo que concierne a aspectos estructurales como son la cantidad recibida, falta de oportunidades laborales o aspectos de índole individual como son la falta de habilidades y capacitación necesaria para acceder a un empleo estable y de mayor remuneración.
\end{enumerate}

\section{Elaboración de Hipótesis}

\begin{itemize}
	\item[-] \textbf{Hipótesis 1:} El aumento de los planes sociales significa mayor dependencia estatal de los más vulnerables.
	\item[-] \textbf{Hipótesis 2:} Un plan social genera una dependencia prolongada del beneficiario la cual puede ser evitada desarrollando programas de inserción laboral que estimulen actitudes de autosuficiencia.
	\item[-] \textbf{Hipótesis 3:} La prolongación de un plan social genera problemas económicos en el estado.
\end{itemize}

\pagebreak

\section{Bibliografía}

\begin{itemize}
	\item[-] Uruguay. Ley N° 18250. \textit{Ley de Migraciones}. (2008). Disponible en \url{https://www.impo.com.uy/bases/leyes/18250-2008}
    \item[-] Uruguay. Ley Nº 18227. \textit{Nuevo Sistema de Asignaciones Familiares a Menores en Situación de Vulnerabilidad}. (2007). Disponible en \url{https://www.impo.com.uy/bases/leyes/18227-2007}
    \item[-] Ministerio de Desarrollo Social (s.f.). \textit{Tarjeta Uruguay Social (TUS)}. \url{https://www.gub.uy/ministerio-desarrollo-social/politicas-y-gestion/programas/tarjeta-uruguay-social}
    \item[-] Ministerio de Desarrollo Social (s.f.). \textit{Uruguay Trabaja}. \url{https://www.gub.uy/ministerio-desarrollo-social/politicas-y-gestion/programas/uruguay-trabaja}
    \item[-] Cecchini, S., Villatoro, P., \& Mancero, X. (2021). \textit{El impacto de las transferencias monetarias no contributivas sobre la pobreza en América Latina}. Revista Cepal.
    \item[-] Aguirre, R., \& Ferrari, F. (2014). \textit{La construcción del sistema de cuidados en el Uruguay}. Naciones Unidas, CEPAL.
    \item[-] Olesker, D. (2011). \textit{La reforma social: hacia una nueva matriz de protección social del Uruguay}.
    \item[-] Hansan, J. E. (2011). English poor laws: Historical precedents of tax-supported relief for the poor. Social welfare history project.
\end{itemize}

\end{document}