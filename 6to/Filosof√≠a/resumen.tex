\documentclass{article}
\usepackage[spanish]{babel}
\usepackage{setspace}
\usepackage[a4paper, total={6in, 9in}]{geometry}

\title{Resumen Curso de Filosofía 3°BD}
\author{Guillermo Wajner}
\setlength{\parskip}{10pt}

\begin{document}

\spacing{1.15}
\maketitle

\section{Ética y Teorías Éticas}


\subsection{Adela Cortina: ¿Qué es la ética?}

Al hablar de la moral, sea individual o colectiva, nos referimos al conjunto de normas, reglas, pautas de conducta, valores y principios que regulan las acciones humanas, distinguiendo lo correcto e incorrecto.

La Ética, por su parte, es una rama de la filosofía que reflexiona sobre la moral. Como tal, esta se construye racionalmente empleando rigor conceptual y métodos de análisis y explicación característicos de la disciplina. La ética busca desplegar los conceptos y argumentos necesarios para comprender la dimensión moral de la persona, sin reducirla a -- pero considerando -- sus componentes individuales, sean psicológicos, sociológicos, económicos o de cualquier otro tipo.



\subsubsection{La Ética como Saber Indirectamente Normativo}

El saber descriptivo refiere a la descripción objetiva o factual de cómo son las cosas o cómo funcionan en la realidad. Busca explicar y comprender fenómenos tal como son, sin emitir juicios de valor o establecer normas. Este tipo de saber se basa en la observación, recopilación de datos y análisis para ofrecer una representación precisa de los hechos.

Por otro lado, el saber normativo se refiere a la formulación de normas, reglas o principios que establecen cómo deberían ser las cosas o cómo deberíamos actuar. Este tipo de saber implica juicios de valor y está orientado hacia la recomendación de comportamientos adecuados, éticos o legítimos. Se basa en sistemas de valores para establecer estándares y pautas de conducta.

Considerando la definición previa de moral, es posible decir que la moral es un saber directamente normativo, pues simplemente establece normas para regular las acciones humanas. La ética no establece normas, si no que racionaliza las establecidas por la moral. 

De esta forma, la ética influye en la moral al plantear preguntas y dilemas éticos, al desafiar y revisar las normas morales establecidas y al proporcionar principios y argumentos racionales para orientar la conducta moral. Por esta influencia, entonces, es posible decir que la ética es indirectamente normativa.


\subsubsection{La Ética como Saber Práctico}

Aristóteles distingue entre tres tipos de saberes: teóricos, poiéticos y prácticos.

\begin{enumerate}
    \item \textbf{Saberes Teóricos}
    
    Los saberes teóricos son saberes descriptivos, (del griego ``\textit{theorein}'': ver, contemplar) se ocupan de averiguar como las cosas son.

    \item \textbf{Saberes Poiéticos}
    
    Los saberes poiéticos son saberes normativos, (del griego ``\textit{poiein}'': hacer, fabricar, producir) se ocupan de averiguar como las cosas se puede producir.

    \item \textbf{Saberes Prácticos}
    
    Los saberes prácticos son saberes normativos, (del griego ``\textit{praxis}'', quehacer, tarea, negocio) se ocupan de averiguar como algo debe ser.
\end{enumerate}

Estos últimos saberes son aquellos que tratan de orientarnos sobre qué debemos hacer para conducir nuestra vida de un modo bueno y justo, cómo debemos actuar, qué decisión es la más correcta en cada caso concreto para que la propia vida sea buena en su conjunto. Tratan sobre lo que debe haber, sobre lo que debería ser (aunque todavía no sea), sobre lo que sería bueno que sucediera (conforme a alguna concepción del bien humano). Intentan mostrarnos cómo obrar bien, cómo conducirnos adecuadamente en el conjunto de nuestra vida.

En la clasificación aristotélica, los saberes prácticos se agrupaban bajo el rótulo de ``filosofía práctica'', rótulo que abarcaba no solo la Ética (saber práctico encaminado a orientar la toma de decisiones prudentes que nos conduzcan a conseguir una vida buena), sino también la Economía (saber práctico encargado de la buena administración de los bienes de la casa y de la ciudad) y la Política (saber práctico que tiene por objeto el buen gobierno de la polis).


\subsection{Epicuro}

El Helenismo, período comprendido entre el 323 a. C. y el 32 a.C es caracterizado por la decadencia de la polis griega a raíz de las conquistas de Alejandro Magno. Las polis en aquel momento eran una parte integra de la identidad personal de los griegos. Esta decadencia trae con si menor participación política e incertidumbre, y surge en la gente un cuestionamiento sobre la importancia de vivir ante las vicisitudes de la realidad. 

Epicuro de Samos, habiendo vivido aproximadamente entre el 341 a. C. y el 270 a. C., vive de primera mano el período helenistico. De esta forma su filosofía, el Epicureísmo, busca encontrr mecanismos para recobrar la felicidad.

El sistema filosófico epicúreo se caracteriza por ser desde un punto de vista canónico, empirista, desde uno físico, atomista, y desde uno ético, hedonista.

El Hedonismo (del griego ``\textit{hedoné}'': placer) plantea que la felicidad radica en el placer. Dentro del Hedonismo, el Epicureísmo plantea una versión moderada de esta corriente, donde la felicidad no es solo el placer, si no la ausencia de dolor. Esto lo hace de forma opuesta a escuelas hedonistas del momento como el cirenaísmo, que planteaban el placer del momento como finalidad universal.

\subsubsection{Hedonismo Epicúreo}

Epicuro considera a la filosofía un camino hacia la felicidad y consideraba que todos merecemos ser felices, así invita a todos a filosofar, sin límites de edad. Él rechaza la idea de que la filosofía es cuestión de madurez, si no una  ``actividad'' encaminada a curar las enfermedades del alma.

Epicuro dice que buscamos placer solo en el momento que nos encontramos en dolor por no tener el mismo, y cuando no sentimos este dolor, ya no necesitamos buscar placer. De esta forma argumenta que el placer es necesario para una buena vida, y que todo placer es cosa buena. Sin embargo, esto no significa que todo placer sea aceptable.

Epicuro distingue entre tres tipos de deseos, los deseos naturales y necesarios, naturales y no necesarios, y vanos.

\begin{enumerate}
    \item \textbf{Deseos Naturales y Necesarios}
    
    Los deseos naturales y necesarios son aquellos vinculados a la supervivencia. Su satisfacción entonces, que es condición necesaria de la felicidad, produce un ``placer catastemático'', fácil de alcanzar, el cual lleva a la ``aponía''(ausencia de turbación corporal). Este tipo de deseo es requerido.

    \item \textbf{Deseos Naturales y no Necesarios}
    
    Los deseos naturales y no necesarios son variaciones de los anteriores. Su satisfacción, sin embargo, produce un placer distinto, el ``placer catastemático'', el cual produce ``aponía'' y ``ataraxia'' (ausencia de turbación espiritual). Este tipo de deseo debe ser moderado para evitar someterse a la posibilidad perpetua de no poder cumplirlos, y para no distraerse de los deseos realmente necesarios, pues pueden aparentar serlo para la felicidad.

    \item \textbf{Deseos Vanos}
    
    Los deseos vanos son aquellos impulsados por honores, glorias y triunfos políticos, la ambición desmedida o presiones externas, y no pueden ser realmente satisfechos. Deseos como el poder o la riqueza son inadquiribles, pues nunca se alcanza la posición de ``el más rico'' o ``el más poderoso''. Epicuro plantea, entonces, que el ``sabio'' debe ser autárquico, independiente económica y psíquicamente, participe en la sociedad, pero alejado de la política. Estos deseos deben evidentemente ser evitados.
\end{enumerate}

\subsubsection{El Tetrafármaco}

El ``tetrafármaco'' o ``cuádruple remedio'' para el alma de Epicuro es otra herramienta, además de la satisfacción de los deseos naturales y no necesarios, para alcanzar la ataraxia. Esta medicina que plantea consiste en evitar temerles a cuatro grandes conceptos: los dioses, la muerte, el dolor, y el futuro.

\begin{enumerate}
    \item \textbf{Los Dioses}
    
    Epicuro toma una postura teísta no intervencionista, así que no se debe temer su castigo, pues no existe el mismo. Los Dioses deben ser considerados ejemplos de felicidad y amistad, así rendirles culto, mientras no sea en exceso produce ataraxia.

    \item \textbf{La Muerte}
    
    Epicuro considera que la muerte del sujeto es la privación de su sensibilidad. Entonces la muerte no es nada para uno, pues cuando la muerte es, el sujeto ya no. Además, consideraba que el alma muere con el cuerpo, y así la muerte no afecta ni a los vivos ni a los muertos. La muerte hace también temer por su inminencia, sin embargo, sabiendo que no es nada para nosotros, Epicuro plantea que su inminencia aporta a la vida feliz, pues elimina el deseo de inmortalidad.

    \item \textbf{El Dolor}
    
    Epicuro plantea que existen dos posibilidades para los dolores. Si un dolor es soportable por el cuerpo, esto significa que el mismo es simplemente pasajero y así no debe ser temido. Por otro lado, si un dolor no es soportable por el cuerpo, es porque este produce la muerte, y por lo descrito anteriormente, tampoco debe ser temido.

    \item \textbf{El Futuro}
    
    Epicuro plantea que el futuro no está ni totalmente determinado ni dentro de nuestro control. Por este motivo, uno no debe ni sumirse en la idea de que el futuro caerá inevitablemente sobre él ni preocuparse por algo que no puede dirigir.
\end{enumerate}

\subsubsection{Los Ingredientes de la Felicidad}

Epicuro plantea, que a pesar de que puede parecer difícil obtener la verdadera felicidad, esta no requiere más que un par de simples componentes. Así, al llegar a Atenas en el año 306 a. C., compra una casa en las afueras de la ciudad, un lugar que se volvería conocido como el ``Jardín''. Este lugar era suficientemente grande como para darle privacidad a varias personas, y con espacios comunes donde se pudiera conversar y comer en conjunto; entonces, él invita a algunos amigos a vivir aquí. Epicuro concuerda con la noción general de que los amigos traen felicidad; sin embargo va más allá y plantea que estos amigos, para realmente traer felicidad, deben ser \textit{\textbf{compañeros permanentes}}. Tal importancia otorga Epicuro a las amistades, que plantea que antes que pensar que uno va a comer, se debe pensar con quien uno va a comer.

Como segundo ingrediente de la felicidad, Epicuro presenta la \textit{\textbf{libertad}}. Para alcanzarla, Epicuro y sus compañeros abandonan Atenas y su atmósfera competitiva por completo, pues ser libre implicaba ser financieramente independientes, económicamente autosuficientes y no responsables ante jefes tiránicos para un sueldo (autárquicos). Fuera de Atenas, Epicuro y sus compañeros fundan lo que sería una comunidad alternativa, una comunidad libre. Llevaban a cabo aquí una vida simple pero gozando de su libertad, independiente de opiniones ajenas.

Por último, Epicuro plantea que existe un tercer ingrediente para la felicidad, el \textit{\textbf{análisis mesurado}}. Epicuro dice que en la vida se debe reservar tiempo para la reflexión, para el análisis de las preocupaciones. Nuestras ansiedades disminuyen rá    pidamente si nos permitimos realmente pensar en ellas.

\subsubsection{Epícuro en el Contexto Actual}

En la actualidad, la sociedad de consumo se ha olvidado de los planteamientos de la moderación de los deseos naturales y no necesarios, olvido reflejado en nuestra incesante búsqueda de nuevos productos. Así también hemos olvidado la importancia de las amistades como compañerismos permanentes, pues hoy consideramos amigos hasta aquel con quien no hablamos más que por medios digitales. Y respecto a la libertad también es evidente que, frente a una sociedad cada vez más global, nos alejamos de la adorada autarquía.

Las publicidades utilizadas en la sociedad consumista actual para vender al público son medios distractores. Nos distraen de nuestros deseos necesarios, haciéndonos creer que los productos que cuya venta incitan los son. Esto lo logran por medio de vínculos sutiles, apelando a, y tomando provecho de estos deseos tan profundos y esenciales del ser humano. Las publicidades difuminan la línea entre nuestros deseos y logran confundirnos.

Entonces la filosofía epicúrea quizás tenga más relevancia de la esperada hoy en día. El Epicureísmo deja mucho que aprender y reflexionar sobre como hoy, más allá de tener todas las posibilidades económicas, la sociedad no nos provee los verdaderos ingredientes de la felicidad. Pues ``pobre no es el que tiene poco, pobre es el que desea infinitamente mucho''. 
\end{document}