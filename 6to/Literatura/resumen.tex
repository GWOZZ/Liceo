\documentclass{article}
\usepackage[spanish]{babel}
\usepackage{bm}
\usepackage{setspace}
\usepackage[a4paper, total={6in, 9in}]{geometry}

\title{Resumen Curso de Literatura 3°BD}
\author{Guillermo Wajner}
\setlength{\parskip}{10pt}

\begin{document}

\spacing{1.15}
\maketitle

\section{La Ilustración (1715 - 1789)}

\subsection{``¿Qué es La Ilustración?''}

Kant define la Ilustración como el proceso de abandono del estado de minoría autoimpuesto del hombre en su famoso ensayo. Plantea que el estado de minoría es producto de pereza y cobardía. El hombre no tiene el valor de pensar por si mismo, subjugandose a los dominios de razón de alguna autoridad.

Kant considera este proceso de salida algo obligatorio, contraponiendo el uso de la razón privado y público. El uso privado de la razón se da cuando el hombre es pieza de una sistema, cuando tiene un papel que representar en la sociedad y funciones que ejercer. Se encuentra en una posición definida, donde debe aplicar unas reglas y perseguir fines particulares, es un uso restringido. De forma contraria, el uso público de la razón se da cuando el hombre razona solo para ser razonable, de forma libre.

\subsection{Voltaire y Leibniz}

\subsubsection*{Voltaire}

Durante la Ilustración, los autores como Voltaire se enfocaron en difundir el valor intelectual y el uso de la razón en la sociedad, promoviendo el pensamiento crítico y la libertad de expresión en oposición a la obediencia ciega a la autoridad. De esta forma, fue crítico de la religión organizada y se opuso al fanatismo religioso, que consideraba como una fuente de violencia e intolerancia. Creía en una religión racional, o deísmo, que rechazaba la idea de un Dios personal que intervenía en el mundo y en su lugar veía a Dios como una fuerza distante y pasiva que creó el universo y dotó a la humanidad de razón y moralidad. Veía la religión como útil para la cohesión social, pero rechazaba la autoridad de la Iglesia y criticaba su explotación de la credulidad del pueblo como método de mantener el poder.

La Iglesia católica tenía un gran poder en la Francia del S. XVIII, y su estrecha relación con la monarquía significaba que las críticas a la Iglesia también se consideraban amenaza para el poder y el orden establecido de la monarquía francesa de la época. Las críticas de Voltaire a la Iglesia católica eran conocidas, y sus ideas disruptivas y religioso-disidentes llevaron a la censura de su libro “Cartas inglesas” y a casi acabar en la cárcel. Por esto, se vio forzado a publicar obras en otros países, escapando de la censura francesa.

A pesar de que la frase ``no estoy de acuerdo con lo que usted dice, pero defenderé a muerte su derecho a decirlo'' es frecuentemente atribuída a Voltaire, esta en realidad fue originalmente redactada en el libro ``Los Amigos de Voltaire'' de Evelyn Beatrice Hall. Dejando esto de lado, la cita igualmente refleja una idea central dentro de la filosofía de Voltaire. Este creía que para que una sociedad sea libre y justa, la tolerancia religiosa y filosófica es una necesidad, esencial para el progreso y el desarrollo de la humanidad. Voltaire plantea que la tolerancia no implica renunciar a las propias creencias, ni aceptar todas las creencias por igual, sino respetar el derecho de cada persona a tener sus propias creencias y prácticas religiosas o filosóficas, siempre y cuando no dañen a los demás.

\subsubsection*{Leibniz}

El filósofo alemán Leibniz plantea que el mundo fue creado de la mejor forma posible por Dios. De esta afirmación surge la siguiente pregunta: ¿por qué, si Dios es bueno, existe el sufrimiento y el mal en el mundo? Leibniz explica que todos los males del mundo contribuyen a un bien mayor, que todo mal parcial es parte de un bien universal. El mal que observamos, en realidad no existe, es un producto de nuestra visión limitada del esquema total. Esta filosofía llamada \textit{optimismo Leibniziano} fue una idea por lo general aceptada en el S. XVIII.

\subsection{``Cándido, o El Optimismo''}

Voltaire consideraba que los planteamientos de Leibniz eran ridículos. Entonces, para demostrar la absurdez de los mismos, Voltaire escribe ``Cándido, o El Optimismo'', y a pesar de su aceptción general, la pone en esta sátira, desde el maestro Pangloss y su discípulo Cándido.

La sátira es un género literario o artístico que utiliza el humor, la ironía y la exageración para ridiculizar o criticar vicios, defectos o aspectos negativos de la sociedad, la política, la religión, la cultura o la moral. La sátira puede ser utilizada como una herramienta para provocar la reflexión y el cambio social, ya que a través del uso del humor y la exageración puede mostrar de manera más evidente las incongruencias y contradicciones de la sociedad y sus instituciones.

En el epígrafe, para evitar posible persecución, Voltaire dice que fue escrito por un ficticio ``Doctor Ralph'', ya muerto en 1759.

La clasificación de Cándido resulta difícil desde el inicio. El  texto puede desorientar en su variedad, pero esta está justificada por la necesaria libertad de escritura, así como por la necesidad de multiplicar los puntos de vista narrativos con el fin de que la ficción se ponga al servicio del siglo de las luces: criticar la realidad desde la escritura.

En el S. XVIII, un cuento siempre partía de lo inexplicable o lo fantástico. Por otro lado, la novela pretende reflejar la realidad, con cierta función didáctica. Paradojicamente, parecería que ambas clasificaciones aplican a Cándido.

La obra puede ser clasificada entonces tanto como cuento largo, como como novela corta, escapando de clasificaciones únicas. Dentro de estas categorías, esta puede pertenecer a cualquiera de los siguientes subgéneros: novela educativa, novela de aventuras, novela de vagabundeo, o cuento filosófico o de tesis.

Sintetizando, las distintas calsificaciones de este texto son: 

\begin{enumerate}
    \item \textbf{Novela}
    \begin{itemize}
        \item[-] \textbf{Novela Educativa}: la evolución o progreso de un personaje a lo largo de la historia lo lleva de la ingenuidad a la madurez.
        \item[-] \textbf{Novela de Aventuras}: se exalta la figura del héroe que logra atravesar obstáculos de manera victoriosa, basada en el gancho psicológico (curiosidad del lector por la resolución del conflicto)
        \item[-] \textbf{Novela de Vagabundeo}: el personaje principal se mueve de un lado a otro.
    \end{itemize} 
    \item \textbf{Cuento}
    \begin{itemize}
        \item[-] \textbf{Cuento Filosófico o de Tesis}: la narración está al servicio de construir o destruir (como es el caso) una postura filosófica o argumento.
    \end{itemize} 
\end{enumerate}

\subsection{La Utopía}

Una utopía, como lo indica el nombre, es un \textit{no lugar}, un proyecto irrealizable. Los lugares utópicos (nótese el aburdo), no tienen un referente en la realidad específico, tienen una geografía incierta, la cual suele ser compensada en el relato por la abundancia de detalles sobre la vida de  quienes viven allí.

Para que un lugar sea considerado utopía, Fernando Aínsa reconoce cinco puntos:

\begin{itemize}
    \item[-] \textbf{Insularidad}: se necesita del aislamiento geográfico para evitar cualquier contaminación exterior.
    \item[-] \textbf{Autarquía}: se gobierna y abastece por sí mismo.
    \item[-] \textbf{Acronía}: existe una ausencia de dimensión histórica, por lo que hay una sensación de presente eterno.
    \item[-] \textbf{Planificación urbanística}: los lugares son diagramados bajo el modelo griego (regular y geométrico).
    \item[-] \textbf{Reglamentación}: la utopía se ve reglamentada (determinadas las reglas por sus habitantes).
\end{itemize}

\section{El Sturm und Drang (1776 - 1784)}

\subsection{¿Qué es El Sturm und Drang?}

El Sturm und Drang fue un movimiento literario y artístico juvenil. Surge en el período post-Ilustración y pre-romanticismo en Alemania, expandiendose a Francia, pero no tomando el resto de Europa. A pesar de ser concebido como pasajero en su tiempo por su carácter juvenil, hoy día es reconcido como un movimiento literario significante.

Las principales características del movimiento eran:
\begin{itemize}
    \item[-] \textbf{Rechazo al Racionalismo}: No se trata de desvalorizar la razón como instrumento del conocimiento humano, sino de exaltar la actitud de la personalidad impulsiva que actúa guiada por la pasión antes que por la reflexión racional.
    \item[-] \textbf{Valoración de lo Misterioso}: Un componente donde los hechos ocurren ante la ausencia de la lógica racional. Los autores acuden a la fuente de las leyendas y supersticiones populares para ello.
    \item[-] \textbf{Exaltación del Sentimiento}: En oposición al
    rechazo que proponía la Ilustración, el impulso emanado de los sentimientos y las emociones (especialmente del amor y la pasión) es valorado como factor predominante al que es preciso liberar.
    \item[-]  \textbf{Exaltación de lo Individual}: El impulso creador del artista es visto como una manifestación de la individualidad a través de sus sensaciones, su inspiración, las visiones de la intuición y la influencia del amor. Lo esencial de la escritura consiste en una emancipación del espíritu, una especie de confesión íntima, que bucea en lo más singular del individuo
\end{itemize}

\subsection{La Novela Epistolar}

La carta es más que un medio de transmisión a distancia de un mensaje. En su forma más artística, la carta se convierte en un género literario con grandes potencialidades expresivas. La Novela Epistolar parte de la comunicación prototípica que establece una carta, pero con el pre-romanticismo, la carta sirve para el autoanálisis, confidencialismo y confesionalismo. La característica principal de este género es la discontinuidad de las ideas, producidas de por los lapsos temporales entre carta y carta. Los huecos y silencios son relevantes, despertando en el receptor la necesidad de completar las lagunas. El autor nos presenta una duración concreta, documentable en las fechas de las cartas, reflejando así el paso del tiempo.

\subsection{``Las Cuitas del Joven Werther''}

El joven Goethe, uno de los principales autores del movimiento, escribe esta primera obra a los 24 años. La historia es bastante autobiográfica: plasma la pasión de Goethe por Carlota. El formato está constituido por cartas que un hombre joven, Werther, manda a un fiel amigo. Estas cartas incluyen muchas reflexiones sobre la sociedad en que vive Werther, sus opiniones y su manera de ver al mundo. Todo siempre pasa por la lente de su estado de ánimo.

\section{Romanticismo (1790 $\bm{\sim}$ 1850)}

El Romanticismo como movimiento artístico surge a modo de reacción frente a la Ilustración. Las obras del movimiento se destacan por la \textbf{exaltación del sentimiento} y la \textbf{libertad formal}. De este modo, las obras literarias del movimiento resultan \textbf{accesibles} a la persona promedio de la época, a medio camino entre el refinaminto cultural y la casi total ignorancia de gran parte de la población.

Como sucede también en movimientos anteriores, el Romanticismo intenta aproximarse a las expresiones populares. Entonces, el estilo característico de la literatura romántica es la invocación de los sentimientos, especialmente los de índole \textbf{individual} y \textbf{subjetiva}, como puede ser el amor.

Fundamentalmente, en los textos del movimiento destaca el predominio ---y frecuente triunfo--- del sentimiento, la emoción y la intuición, sobre la razón, la lógica y la certidumbre de la ciencia.

Otras características destacables del movimiento son las siguientes:

\begin{itemize}
    \item[-] \textbf{Individualismo Subjetivo}: La percepción de la
    realidad se concibe en términos de aceptación o rechazo (generalmente de rechazo). Este subjetivismo se expresa a través de temas recurrentes como el amor no correspondido, el sentido de frustración, la soledad, la tristeza, la nostalgia y la desesperación.
    \item[-] \textbf{Frustración del Individuo}: La persona forma una actitud de rebeldía irraciconal, a causa de la frutración por el contraste entre este y la sociedad.
    \item[-] \textbf{Naturalismo}: Se tiende representar la vida en un ambiente no contaminado por el hombre, donde la serenidad del ambiente lleva a la introspección. Esta visión dramática y sentimental de la naturaleza lleva al uso frecuente del paralelismo psicocósmico.
    \item[-]  \textbf{Esoterismo}: El artista no se siente cómodo en la sociedad y realidad actual, de las cuales pretende escapar. Así, toma una actitud donde se aleja de estas, situándose en tiempos pasados.
\end{itemize}

\subsection{El Mal del Siglo}

``El Mal del Siglo'' es un tópico utilizado para referir al movimiento del Romanticismo. El movimiento conlleva una decadencia y crisis de los valores y creencias obtenidos bajo movimientos anteriores. Esto produce un malestar existencial, también causado posiblemente por el abandono del racionalismo típico de la Ilustración.

Esto considerado, los jóvenes románticos quienes quisieron revolucionar el arte de un mundo que los cansaba y angustiaba, lo lograron. Sin embargo, hoy son vistos por algunos como una juventud enferma, desencantada, pesimista, frustrada, melancólica, vencida, solitaria, insatisfecha, dolorosa y tendiente a la locura, al
suicidio, o a la muerte temprana.

\subsection{La Modernidad}

En un artículo escrito por Charles Baudelaire en 1859, este introduce la noción de "modernidad". Baudelaire concibe a modernidad como una cualidad el hombre de la época. La modernidad implica la capacidad de dentro del entorno urbano y decadente, encontrar una belleza misteriosa y hasta entonces desconocida.

La obra de Baudelaire problematiza la existencia de la poesía en un mundo dominado por la utilidad y la obsesión por el progreso. Este se opone a la trivialidad positiva y al progreso meramente material, que limitan las aspiraciones espirituales y sofocan la poesía.

Su poesía ---o la poesía moderna en general---, surgen de la posición del hombre en medio de la nueva civilización. Esta posición produce en el sujeto un profundo y creciente sentimiento de soledad, una sensación de inseguridad oculta tras la confianza en el progreso, y una creciente deshumanización en un mundo dominado por la artificialidad y separado de la naturaleza.

A pesar de esto, es hombre encuentra una fascinación por una atracción misteriosa producida por la urbe. Esto abre la posibilidad al poeta de convertir en arte la misma causa de su angustia. Según Budelaire, el poeta debe extraer lo eterno, de lo transitorio.

\subsection{El Simbolismo}

El simbolismo es un movimiento artístico que se caracteriza por sugerir ideas o evocar objetos a travez de símbolos o imágenes. El Simbolismo entonces impulsa la creación basada en la idea, dejando de lado el objeto no nombrándolo de forma directa.

Dentro de la literatura, el Simbolismo corresponde a la búsqueda de un nuevo lenguaje basado en la sugerencia y la ausencia de la claridad definitoria. Los mensajes transmitidos deben ser descubiertos e interpretados por el lector, siempre a través de una atmósfera evocativa. Así, el lector sensible comprende las alusiones y vive en el mundo simbólico creado por el poeta.

\subsection{El Romanticismo Oscuro}

El Romanticismo Oscuro es un subgénero literario que surge a partir del movimiento romántico. Mientras el Romanticismo en general se caracterizaba por celebrar la euforia y la sublimidad, los autores del Romanticismo Oscuro se apartan de esta visión optimista y tienden a tratar temas más sombríos y oscuros.

Textos pertenecientes al Romanticismo Oscuro muestran las tendencias de sus autores desde el crimen, la atmósfera grotesca, lo irracional y la aparición de fantasmas. El subgénero tiene como objetivo llevar al lector al miedo y la inquietud, poniéndolo tanto frente a los temores más universales, como a miedos particulares de su propia época.

\subsubsection*{El Cuento de Terror}

El cuento de terror es un subgénero del género narrativo, el cual se distingue por tres principales rasgos: la \textbf{atmósfera}, la \textbf{gradación} y el \textbf{mal}. En primer lugar, se tiende a cuidar el clima o la atmósfera que rodea los acontecimientos. La narración presenta una estructura lenta y en gradual ascendencia, hasta alcanzar un máximo en el climax. Por último, el texto no presenta una postura moral sobre los personajes y sus cometidos.

\subsection{La Relación Poe-Baudelaire}

El desarrollo literario de Edgar Allan Poe precedió al de Charles Baudelaire, pero a pesar de la distancia temporal y geográfica, ambos autores mantuvieron una estrecha relación. Baudelaire admiraba profundamente a Poe y encontraba en sus obras temas e ideas que parecían surgir de su propia mente. Sus escritos compartían una estética sombría y exploraban la oscuridad y la condición humana. Baudelaire dedicó quince años a traducir las obras de Poe al francés, lo que contribuyó a una mayor difusión internacional de la obra de Poe, y su influencia se extendió incluso a otros traductores como Julio Cortázar. 

\subsection{"Las Flores del Mal"}

``Las flores'' del mal es una colección de poemas icónica de Baudelaire, considerada su obra maestra. Surgida en la época del Romanticismo, destaca por abordar los grandes conflictos de este movimiento de manera intensa, llevándolos al máximo grado. Aunque comparte temas importantes con la poesía romántica, como la concepción del poeta como ser excepcional y solitario, el impulso hacia el sueño y las fuerzas irracionales, y la valoración de lo subjetivo y la emoción sobre la racionalidad, se diferencia al transformar radicalmente estos temas en su obra.

\end{document}